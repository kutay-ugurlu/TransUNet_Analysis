\documentclass{IEEEtran}
\usepackage{mathtools}
\usepackage{graphicx}
\usepackage{amssymb}
\usepackage{amsmath}
\usepackage{pythonhighlight}
\usepackage[utf8]{inputenc}
\usepackage{fancyhdr}
\usepackage{pythonhighlight}
\usepackage{changepage}
\usepackage{slashbox}
\usepackage{floatrow}
\usepackage{listings}
\usepackage{derivative}
\usepackage[hidelinks]{hyperref}
\usepackage{fontawesome}
\usepackage{caption}
\usepackage{subcaption}
\usepackage[sorting=none,style=ieee]{biblatex}
\usepackage{cleveref}
\usepackage{algorithm}
\usepackage{algpseudocode}
\usepackage{physics}
\usepackage{lipsum}

\addbibresource{report.bib}
\title{Analysis of TransUnet}
\author{Kutay Ugurlu}

\begin{document}
\maketitle
\begin{abstract}
    The electrical behaviors of specific kinds of organs and structures in the human body provide crucial information about the workings of physiological-level structures and mechanisms of certain diseases. For diagnosis regarding such structures, many imaging methods have been developed exploiting these behaviors. To do so, a good understanding of electrical cell behavior is required. This report provides a theoretical background for the action potential behavior, using Hodgkin-Huxley's explanation, and explains the methodology to create a simulation software that illustrates the generation and propagation of the action potentials. The developed code is available online at \href{https://github.com/kutay-ugurlu/Hodgkin-Huxley-Membrane-Model}{https://github.com/kutay-ugurlu/Hodgkin-Huxley-Membrane-Model}
\end{abstract}
\section{Intro}


\printbibliography{}
\end{document}

